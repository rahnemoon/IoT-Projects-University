%%%%%%%%%%%%%%%%%%%%%%%%%%%%%%%%%%%%%%%%%%%%%%%
%%% Template for lab reports
%%%%%%%%%%%%%%%%%%%%%%%%%%%%%%%%%%%%%%%%%%%%%%%

%%%%%%%%%%%%%%%%%%%%%%%%%%%%%% Sets the document class for the document
% Openany is added to remove the book style of starting every new chapter on an odd page (not needed for reports)
\documentclass[10pt,english, openany]{book}

%%%%%%%%%%%%%%%%%%%%%%%%%%%%%% Loading packages that alter the style
\usepackage[]{graphicx}
\usepackage[]{color}
\usepackage{alltt}
\usepackage[T1]{fontenc}
\usepackage[utf8]{inputenc}
\setcounter{secnumdepth}{3}
\setcounter{tocdepth}{3}
\setlength{\parskip}{\smallskipamount}
\setlength{\parindent}{0pt}

% Set page margins
\usepackage[top=100pt,bottom=100pt,left=68pt,right=66pt]{geometry}

% Package used for placeholder text
\usepackage{lipsum}

% Prevents LaTeX from filling out a page to the bottom
\raggedbottom

% Adding both languages
\usepackage[english, italian]{babel}

% All page numbers positioned at the bottom of the page
\usepackage{fancyhdr}
\fancyhf{} % clear all header and footers
\fancyfoot[C]{\thepage}
\renewcommand{\headrulewidth}{0pt} % remove the header rule
\pagestyle{fancy}

% Changes the style of chapter headings
\usepackage{titlesec}
\titleformat{\chapter}
   {\normalfont\LARGE\bfseries}{\thechapter.}{1em}{}
% Change distance between chapter header and text
\titlespacing{\chapter}{0pt}{50pt}{2\baselineskip}

% Adds table captions above the table per default
\usepackage{float}
\floatstyle{plaintop}
\restylefloat{table}

% Adds space between caption and table
\usepackage[tableposition=top]{caption}

% Adds hyperlinks to references and ToC
\usepackage{hyperref}
\hypersetup{hidelinks,linkcolor = black} % Changes the link color to black and hides the hideous red border that usually is created

% If multiple images are to be added, a folder (path) with all the images can be added here 
\graphicspath{ {Figures/} }

% Separates the first part of the report/thesis in Roman numerals
\frontmatter


%%%%%%%%%%%%%%%%%%%%%%%%%%%%%% Starts the document
\begin{document}

%%% Selects the language to be used for the first couple of pages
\selectlanguage{english}

%%%%% Adds the title page
\begin{titlepage}
	\clearpage\thispagestyle{empty}
	\centering
	\vspace{1cm}

	% Titles
	% Information about the University
	{\normalsize Internet of Things \\ 
		Computer Science and Engineering \\
		Politecnico di Milano \par}
		\vspace{3cm}
	{\Huge \textbf{TinyOS Second Homework}} \\
	%\vspace{1cm}
	%{\large \textbf{xxxxx} \par}
	\vspace{4cm}
	{\normalsize Erfan Rahnemoon 10720184 - 943057 \par}
	\vspace{5cm}
    
    \centering \includegraphics[scale=0.4]{logo1.pdf}
    
    \vspace{0.5cm}
		
	% Set the date
	{\normalsize 22-03-2020 \par}
	

\end{titlepage}

%%%%%%%%%%%%%%%%%%%%%%%%%%%%%%%%%%%%%%%%%%%%%%%%%%%%%%%%%%%%%%%%%%%%%%%%%%%%%%%%%%%%%%%%%%%%
%%%%%%%%%%%%%%%%%%%%%%%%%%%%%%%%%%%%%%%%%%%%%%%%%%%%%%%%%%%%%%%%%%%%%%%%%%%%%%%%%%%%%%%%%%%%
%%%%% Text body starts here!
%\mainmatter

\chapter{Summary}\label{chapt:sum}
[\textit{An application which two node communicate with each other by request/response diagram which the both request and reponse massages are acknowledged.}]
\chapter{Description of the Implementation}
First of all, for the transmitted data a sensor in simulated in the "sensorC" and "sensorAppC" files in which the data for simulation is coming from the DATA directory, and this component will provide an interface to be used in the main program. In the main program depending on the node ID, a node will send a request or response, which for both a separated function is defined. The rest of the code is followed the Acknowledgment draft in TinyOS with one exception. In the "sendDone" function, if a request packet is not acknowledged the timer for the request will be postponed to twice as the current period of the timer, and if the acknowledgment fails this trend will continue until the timer delay is 20 second. However, if the acknowledgment was successful the delay time will be halved until the delay time is one second, again. Also in the same function, if the acknowledgment fails based on the ID of the node the request or response send will be repeated. In the simulation there is a bi-directional link between node zero and one and the node one will start to work five-second later.
Implementation is available in Github\footnote{\url{https://github.com/rahnemoon/IoT-Projects-University}}.\par
Finally, this kind of acknowledgment is synchronized acknowledgment which there is no queue in the implementation, and because the exercise was based on the higher level; the implementation is done in the same layer instead of going to link-layer an implementing the queue and asynchronous acknowledgment. Also, the project is based on the master branch of the tiny os in the Github\footnote{\url{https://github.com/tinyos/tinyos-main}} and using the other version, because of the incompatibility in the toolchain will cause compile errors or warnings.



\end{document}
